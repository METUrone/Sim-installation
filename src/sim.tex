\documentclass{article}
\usepackage{verbatim}
\usepackage[legalpaper,landscape, margin=0.5in]{geometry}

\makeatletter
\newenvironment{mycode}
 {\def\@xobeysp{\ }\verbatim\rightskip=0pt plus 6em\relax}
 {\endverbatim}
\makeatother

\title{Installation Documentation}

\begin{document}

\maketitle

\section*{}
\subsection*{}
First, copy ubuntu\_sim\_ros\_melodic.sh to your home directory. \\
Then, open your terminal and call the command \\
\begin{verbatim}
    source ubuntu_sim_ros_melodic.sh
\end{verbatim}
\noindent
This command may take a few minutes, depending on internet connection. It downloads and installs ros, mavros, mavlink, Gazebo, and builds necessary things. \\
After that, in your home directory, there should be a directory called "Firmware". \\
\noindent
Then, call these commands in your home directory. \\
\begin{verbatim}
    git clone https://github.com/METUrone/Firmware.git
    sudo apt install python3-pip
    pip3 install empy toml numpy packaging jinja2
    sudo apt install libgstreamer1.0-0 gstreamer1.0-plugins-base gstreamer1.0-plugins-good gstreamer1.0-plugins-bad gstreamer1.0-plugins-ugly
    sudo apt install gstreamer1.0-libav gstreamer1.0-doc gstreamer1.0-tools gstreamer1.0-x gstreamer1.0-alsa gstreamer1.0-gl gstreamer1.0-gtk3 gstreamer1.0-qt5 gstreamer1.0-pulseaudio
\end{verbatim}
\noindent
Open a new terminal, and write \\
\begin{verbatim}
    gedit .bashrc
\end{verbatim}
\noindent
To this file's end, paste the following text. \\
\begin{verbatim}
    source Firmware/Tools/setup_gazebo.bash $(pwd)/Firmware $(pwd)/Firmware/build/px4_sitl_default
    export ROS_PACKAGE_PATH=$ROS_PACKAGE_PATH:$(pwd)/Firmware
    export ROS_PACKAGE_PATH=$ROS_PACKAGE_PATH:$(pwd)/Firmware/Tools/sitl_gazebo
\end{verbatim}
\noindent
In the end, you should save and close the file. The last step in the installation is, calling \\
\begin{verbatim}
    gedit ./.ignition/fuel/config.yaml
\end{verbatim}
\noindent
In that file, there is a line that writes \\
\begin{verbatim}
    url: https://api.ignitionfuel.org
\end{verbatim}
\noindent
Change it as the following \\
\begin{verbatim}
    url: https://api.ignitionrobotics.org
\end{verbatim}
\noindent
After that point, all installations are done. \\

For running the simulation, you must run the following commands every time. \\

Now open a new terminal, and call the command \\
\begin{verbatim}
    roscore
\end{verbatim}
\noindent
Then, in a new terminal, call \\
\begin{verbatim}
    cd Firmware
    no_sim=1 make px4_sitl_default gazebo
\end{verbatim}
\noindent
This command will take at least 5 minutes in first call.
Then open a new terminal and call \\
\begin{verbatim}
    roslaunch mavros px4.launch fcu_url:="udp://:14540@127.0.0.1:14557"
\end{verbatim}
\noindent
Again, in a new terminal, call \\
\begin{verbatim}
    roslaunch gazebo_ros empty_world.launch world_name:=$(pwd)/Tools/sitl_gazebo_worlds_iris.world
\end{verbatim}
\noindent
This command will open Gazebo after a few moments. \\
From left, click insert, and from the left menu click 3DR Iris and place it on the simulation. \\
Then, look for the terminals. You should see some changes. \\
To try if it works, open the second terminal and write the command \\
\begin{verbatim}
    commander takeoff
\end{verbatim}
\noindent
This should takeoff the drone on the simulator. \\
After that, you can give your commands using your ros nodes. \\
To try that, download the example package in the repository, build the ros package, and run it while all the terminals and simulation working. \\
The drone should takeoff and stay at 2.5m. \\

\end{document}